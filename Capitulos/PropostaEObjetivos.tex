
\chapter{Proposta e Objetivos do Trabalho}

Neste trabalho abordamos o projeto, análise e implementação do sistema PrISMA – Programa de Instrução à Solicitação de Matrícula Acadêmica. Trata-se de um projeto de pesquisa e desenvolvimento envolvendo o processo de matrícula nas disciplinas da PUC-Rio, que tem por objetivo disponibilizar um sistema de apoio aos alunos de graduação visando otimizar e reduzir os erros na solicitação de matrícula de cada aluno.

O sistema PrISMA lida com alguns problemas bastante relevantes e desafiadores, desde a área de interface humano-computador até a análise de complexidade, passando pela especificação de sistemas de bancos de dados. Por ser um sistema que lida diretamente com um problema que atinge um conjunto de alunos significativo, traz interesse e motivação adicionais para que sejam verificadas soluções computacionais adequadas. Além de formulá-lo conceitualmente, o sistema foi implementado e está em produção, já tendo sido utilizado por quase 1500 alunos no final de 2011. Em função das diversas sugestões recebidas e interesse demonstrado por conta de sua utilidade prática, novas versões do sistema estão sendo discutidas e desenvolvidas.

Vamos discutir neste artigo tanto o problema de decisão – a possibilidade de atender, ou não, uma solicitação de matrícula em disciplinas por completo – como também o problema de otimização – a tentativa de maximizar o conjunto de disciplinas matriculadas em função do conjunto de disciplinas solicitadas. Além disso, são apresentadas também as decisões quanto ao desenvolvimento do sistema PrISMA e sua implementação atual, com versões em PHP, Lua e SGBD MySQL.
