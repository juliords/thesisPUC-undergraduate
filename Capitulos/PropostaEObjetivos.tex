
\chapter{Proposta e Objetivos do Trabalho}

Propõe-se então o sistema PrISMA (Programa de Instrução à Solicitação de Matrícula acadêmica. Trata-se de uma webpage que reune todas as funcionalidades necessárias para auxiliar os alunos no processo de matrícula.

O objetivo do sistema é apresentar uma interface muito poderosa, reunindo todos os dados necessários à solicitação, e ao mesmo tempo intuitiva, de forma que os alunos não fiquem perdidos durante o processo. No entanto, o processo como um todo exige o acesso à muita informação ao mesmo tempo, tornando o desenvolvimento do projeto um grande desafio.

O básico, para os alunos, seria reunir num mesmo ambiente seu Falta Cursar, disciplinas oferecidas pelo Micro Horário e as disciplinas já selecionadas para a grade do próximo período. Assim como uma visualização daquilo de já foi selecionado no formato de tabela de horários semanal. Dessa maneira, não seria mais necessário se perder no meio de diversas aplicações em busca de todas essas informações. Tudo estaria reunido em um único lugar.

Além da preocupação com a nevegação, existe também a preocupação com a corretude da grade gerada. Por especificação todas (ou pelo menos a maioria) das regras que impedem a solicitação de disciplinas devem ser implementadas. Sendo a principal delas a regra de pré-requisitos. Para tal se faz necessário do histórico de disciplinas cursadas dos alunos, e do seu C.R. acumulado. Dessa maneira, conforme o aluno for utilizando e selecionando as disciplinas a serem cursadas, o sistema realiza a verificação e informa o aluno em caso de algum problema.

Buscando diminuir a quantidade de dados apresentados, ao invés de exibir as disciplinas do Falta Cursar do aluno, é précomputado o conjunto de disciplinas chamado de Pode Cursar. Este subconjunto contém apenas as disciplinas que não estão presas por nenhum pré-requisito. 

Olhando do ponto de vista das multiplas opções no campo de solicitação, identifica-se uma grande incompreensão por parte dos alunos dessa funcionalidade no sistema da PUC. Essa falta de clareza se trata de uma das principais fontes de erros dos alunos na hora de preencher a solicitação. Por se tratar de um ponto crítico, propõe-se uma funcionalidade que, passo a passo, mostre ao usuário o que aconteceria caso alguma disciplina, de algum opção fosse rejeitada, e que impacto isso teria no restante da solicitação.

Após terem suas grades montadas, deve-se preencher manualmente no site da PUC-Rio os dados gerados, visto que se trata de um sistema independente. Uma solução para isto seria a definição de um padrão de arquivo exportado pelo PrISMA que fosse aceito pelo PUC Online (setor responsável pelo sistema da PUC-Rio). Entretanto, por se tratar de procedimentos burocráticos, não será de fato implementado pela PUC. Apesar de que o arquivo ainda assim possa ser gerado.

Para aqueles que oferecem as disciplinas é interessante ter dados estatísticos de uso do sistema por parte dos alunos. Apresentações de gráficos de lotação das turmas e como se comportam os alunos ao realizar o procedimento de escolha. Dessa maneira pode-se aprender com a necessidade dos alunos.
