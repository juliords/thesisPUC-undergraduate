
\chapter{Plano de A\c{c}\~ao}

Deve ser destacado que o sistema proposto já foi implementado e usado por cerca de 3000 pessoas nos períodos passados. Dessa maneira, o plano de ação se resume à uma extensa documentação do que foi feito para que, algum dia, alguém possa colocar o projeto para frente utilizando todo o conhecimento adquirido.

Algumas outras funcionalidades como estatísticas de uso e exportação dos dados gerados pelos usuários será implementado na segunda parte do projeto. A primeira é muito interessantes do ponto de vista daqueles que organizam o processo de matrícula. A segunda aos alunos, que não precisariam entrar manualmente com os dados no site da PUC-Rio.

Para que os devidos resultados sejam alcançados será priorizada a documentação. Começando pelo ambiente de desenvolvimento, passando pelo porquê das ferramentas utilizadas até a arquitetura final do sistema.
