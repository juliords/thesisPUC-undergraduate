
\chapter{Introdu\c{c}\~ao}

A Pontifícia Universidade Católica do Rio de Janeiro (PUC-Rio) oferece a seus alunos diversos cursos de graduação, cujas disciplinas são contabilizadas e disponibilizadas em um sistema de contagem de créditos. Os alunos precisam conter em seus históricos escolares um número mínimo de créditos pertencentes a um determinado currículo de um curso para que obtenham seus diplomas. 

Assim, e ao longo de sua vida universitária, os alunos da PUC-Rio buscam organizar os semestres escolares e escolhas de disciplinas de forma que consigam se formar no menor tempo possível. Apenas alunos calouros não fazem escolhas de disciplinas (e respectivos créditos). A partir do segundo período de seu curso, cada aluno da PUC-Rio deve escolher quais disciplinas para o novo período escolar pretende cursar.

As decisões sobre a escolha de disciplinas são baseadas na oferta de disciplinas daquele período, nos pré-requisitos e compatibilidade de horários. Há também de se considerar turmas distintas de mesmas disciplinas, horários conflitantes, limite no total de vagas oferecidas por turma e total de créditos permitido por semestre. Há também preferências (ou impedimentos) de caráter pessoal, como é o caso de alunos que têm outras atividades (e.g. estágio em empresas ou laboratórios acadêmicos) em paralelo com suas vida acadêmica. Cabe observar que a PUC-Rio determina prioridades em relação à ocupação de vagas de acordo com uma ordenação interna que pondera basicamente o desempenho acadêmico do aluno (C.R. – coeficiente de rendimento), a quantidade de créditos já obtidos, consequentemente, o tempo estimado para completar os créditos necessários para concluir o curso.

A PUC-Rio requisita, a cada semestre, que os alunos preparem uma solicitação de matrícula com três opções contendo o conjunto de disciplinas e turmas que pretende cursar no semestre seguinte. Esta escolha é chamada de grade: disciplinas, dias e horários escolhidos em função da lista de turmas oferecidas pela Universidade para o período seguinte. Há restrições de montagem da grade, como por exemplo, um valor máximo de horas (ou créditos) semanais, turma de matérias distintas par a par; e turmas cujos horários não conflitem entre si, ou seja, que tenham os horários disjuntos, também par a par.

Caso a primeira opção de um aluno não seja totalmente atendida, por exemplo, por falta de vaga, o sistema de matrícula tenta incluir as disciplinas escolhidas em segunda ou terceira opção, buscando atender os pedidos de um aluno. É fato que muitos alunos não se dão por satisfeitos pois muitas disciplinas solicitadas ficam de fora da seleção final de disciplinas efetivamente matriculadas inicialmente. Sem o apoio de uma ferramenta adequada, os ajustes de matrícula em disciplinas, realizados a cada início de semestre escolar, se tornaram cada vez mais frequentes. Entre outros motivos, uma das razões que fazem com que os alunos não fiquem satisfeitos com sua grade de disciplinas escolar diz respeito às decisões equivocadas no momento de solicitação e da montagem da grade de opções.

Consequentemente, a PUC-Rio deve alocar mais recursos para se tentar resolver a situação individual de cada aluno pós-solicitação, os alunos precisam comparecer, muitas das vezes, pessoalmente para tentar corrigir suas grades e todo o processo fica mais complicado, e com resultados menos desejáveis. Implicando assim em perda para ambas as partes.

% \begin{figure}
%   \includegraphics*[width=\linewidth]{ctor4_none.eps}
%   \caption{Uma figura}
% \end{figure}
