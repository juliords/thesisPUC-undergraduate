\documentclass[graduacao,brazil]{ThesisPUC}


%%%%%%%%%%%%%%%%%%%%%%%%%%%%%%%%%%%%%%%%%%%%%%%%%%%%%%%%%%%%%%%%%%%%%%%%%%%%%%%%

\newcommand{\Rset}{\mathbb{R}}
\newcommand{\Zset}{\mathbb{Z}}

%%%%%%%%%%%%%%%%%%%%%%%%%%%%%%%%%%%%%%%%%%%%%%%%%%%%%%%%%%%%%%%%%%%%%%%%%%%%%%%%

\autor{Julio Ribeiro da Silva}
\autorR{Calazans, Breno}
\orientador{Gustavo Robichez de Carvalho}
\orientadorR{Carvalho, Gustavo Robichez de}

\titulo{Nome do seu projeto final}
\titulouk{Title of your project}

\subtitulo{Subt\'{i}tulo (opcional)}
\dia{07} \mes{Setembro} \ano{2014}

\cidade{Rio de Janeiro}
\CDD{510}
\departamento{Inform\'atica}
\programa{Sistemas de Informa\c{c}\~{o}es}
\centro{Centro T\'{e}cnico Cient\'{i}fico}
\universidade{Pontif\'{i}cia Universidade Cat\'{o}lica do Rio de Janeiro}
\uni{PUC--Rio}
\course{Sistemas de Informa\c{c}\~ao}
\diploma{Bacharel em Sistemas de Informa\c{c}\~ao}

%%%%%%%%%%%%%%%%%%%%%%%%%%%%%%%%%%%%%%%%%%%%%%%%%%%%%%%%%%%%%%%%%%%%%%%%%%%%%%%%

% Não precisa preencher se for um Projeto Final %
%
\banca{
  \membrodabanca{Luis Carlos Pacheco R. Velho}{IMPA}
  \membrodabanca{Jorge Stolfi}{UNICAMP}
  \coordenador{Ney Augusto Dumont}
}
%

%%%%%%%%%%%%%%%%%%%%%%%%%%%%%%%%%%%%%%%%%%%%%%%%%%%%%%%%%%%%%%%%%%%%%%%%%%%%%%%%

% Não precisa preencher se for um Projeto Final %
%
\curriculo{%
Graduou--se em Engenharia na Ecole Polytechnique (Paris, Fran\c{c}a), cursando \'{A}lgebra e Inform\'{a}tica, assim como F\'{i}sica Te\'{o}rica. Especializou--se na Ecole Sup\'{e}rieure des T\'{e}l\'{e}communications (Paris, Fran\c{c}a) em Processamento de Sinais de Voz e Imagens, assim como Organiza\c{c}\~{a}o e Planejamento. Trabalhou junto com a empresa Inventel em sistemas de telecomunica\c{c}\~{o}es sem fil baseados na tecnologia BlueTooth. Desenvolveu junto com os seus orientadores durante o Mestrado ferramentas de topologia computacional.}%
%


%%%%%%%%%%%%%%%%%%%%%%%%%%%%%%%%%%%%%%%%%%%%%%%%%%%%%%%%%%%%%%%%%%%%%%%%%%%%%%%%

\epigrafe{%
Lorem ipsum.
}
\epigrafeautor{Wassily Kandinsky}
\epigrafelivro{Regarde}

%%%%%%%%%%%%%%%%%%%%%%%%%%%%%%%%%%%%%%%%%%%%%%%%%%%%%%%%%%%%%%%%%%%%%%%%%%%%%%%%

\agradecimentos{%
Aos meus orientadores Professores H\'{e}lio Lopes e Geovan Tavares pelo apoio, simpatia de sempre, e incentivo para a realiza\c{c}\~{a}o deste trabalho

Ao CNPq e \`{a} PUC--Rio, pelos aux\'{i}lios concedidos, sem os quais este trabalho n\~{a}o poderia ter sido realizado.

\`{A}s minhas av\'{o}s, que sofreram o mais pela saudade devida a minha expatria\c{c}\~{a}o. Aos meus pais, irm\~{a}s e fam\'{i}lia.

Aos meus colegas da PUC--Rio, quem me fizeram adorar esse lugar.

Aos professores Marcos da Silvera, Jean--Marie Nicolas e Anne Germa que me ofereceram a oportunidade desta coopera\c{c}\~{a}o.

Ao pessoal do departamento de Matem\'{a}tica para a ajuda de todos os dias, em particular \`{a} Ana Cristina, Creuza e ao Sinesio.
}


%%%%%%%%%%%%%%%%%%%%%%%%%%%%%%%%%%%%%%%%%%%%%%%%%%%%%%%%%%%%%%%%%%%%%%%%%%%%%%%%

\chaves{%
  \chave{Teoria de Morse}%
  \chave{Teoria de Forman}%
  \chave{Topologia Computacional}%
  \chave{Geometria Computacional}%
  \chave{Modelagem Geom\'{e}trica}%
  \chave{Matem\'{a}tica Discreta}%
}

\resumo{
A teoria de Morse \'{e} considerada uma ferramenta matem\'{a}tica importante em aplica\c{c}\~{o}es nas \'{a}reas de topologia computacional, computa\c{c}\~{a}o gr\'{a}fica e modelagem geom\'{e}trica. Ela foi inicialmente formulada para variedades diferenci\'{a}veis. Recentemente, Robin Forman desenvolveu uma vers\~{a}o dessa teoria para estruturas discretas, tais como complexos celulares. E isso permitiu que ela pudesse ser aplicada a outros tipos interessantes de objetos, em particular para malhas.

Uma vez que uma fun\c{c}\~{a}o de Morse \'{e} definida em uma variedade, informa\c{c}\~{o}es sobre sua topologia podem ser deduzidas atrav\'{e}s de seus elementos cr\'{i}ticos. O objetivo desse trabalho \'{e} apresentar um algoritmo para definir uma fun\c{c}\~{a}o de Morse discreta \'{o}tima para um complexo celular, onde obter o \'{o}timo significa construir uma fun\c{c}\~{a}o que possui o menor n\'{u}mero poss\'{i}vel de elementos cr\'{i}ticos. Aqui foi provado que esse problema \'{e} MAX--SNP dif\'{i}cil. Entretanto, tamb\'{e}m ser\'{a} proposto um algoritmo linear que, para o caso de variedades de dimens\~{a}o 2, \'{e} sempre \'{o}timo.

Tamb\'{e}m foram provados v\'{a}rios resultados sobre a pr\'{o}pria estrutura das fun\c{c}\~{o}es de Morse discretas. Em particular, uma representa\c{c}\~{a}o equivalente por hiperflorestas \'{e} apresentada. E atrav\'{e}s dessa representa\c{c}\~{a}o, foi desenvolvido um algoritmo para constru\c{c}\~{a}o de fun\c{c}\~{o}es de Morse discretas em complexos celulares com dimens\~{a}o arbitr\'{a}ria. Esse algoritmo \'{e} quadr\'{a}tico no tempo e, apesar de n\~{a}o se poder garantir o resultado \'{o}timo, d\'{a} respostas \'{o}timas na maioria dos casos pr\'{a}ticos.
}


%%%%%%%%%%%%%%%%%%%%%%%%%%%%%%%%%%%%%%%%%%%%%%%%%%%%%%%%%%%%%%%%%%%%%%%%%%%%%%%%

\chavesuk{
  \chave{Morse Theory}%
  \chave{Forman Theory}%
  \chave{Computational Topology}%
  \chave{Computational Geometry}%
  \chave{Solid Modeling}%
  \chave{Discrete Mathematics}%
}

\resumouk{%
Morse theory has been considered a powerful tool in its applications to computational topology, computer graphics and geometric modeling. It was originally formulated for smooth manifolds. Recently, Robin Forman formulated a version of this theory for discrete structures such as cell complexes. It opens up several categories of interesting objects (particularly meshes) to applications of Morse theory.

Once a Morse function has been defined on a manifold, then information about its topology can be deduced from its critical elements. The purpose of this work is to design an algorithm to define optimal discrete Morse functions on general cell complex, where optimality entails having the least number of critical elements. This problem is proven here to be MAX--SNP hard. However, we provide a linear algorithm that, for the case of 2--manifolds, always reaches optimality.

Moreover, we proved various results on the structure of a discrete Morse function. In particular, we provide an equivalent representation by hyperforests. From this point of view, we designed a construction of discrete Morse functions for general cell complexes of arbitrary finite dimension. The resulting algorithm is quadratic in time and, although not guaranteed to be optimal, gives optimal answers in most of the practical cases.
}


%%%%%%%%%%%%%%%%%%%%%%%%%%%%%%%%%%%%%%%%%%%%%%%%%%%%%%%%%%%%%%%%%%%%%%%%%%%%%%%%

\modotabelas{figtab} % nada, fig, tab ou figtab

%%%%%%%%%%%%%%%%%%%%%%%%%%%%%%%%%%%%%%%%%%%%%%%%%%%%%%%%%%%%%%%%%%%%%%%%%%%%%%%%

\begin{document}


\chapter{Introdu\c{c}\~ao}

A Pontifícia Universidade Católica do Rio de Janeiro (PUC-Rio) oferece a seus alunos diversos cursos de graduação, cujas disciplinas são contabilizadas e disponibilizadas em um sistema de contagem de créditos. Os alunos precisam conter em seus históricos escolares um número mínimo de créditos pertencentes a um determinado currículo de um curso para que obtenham seus diplomas. 

Assim, e ao longo de sua vida universitária, os alunos da PUC-Rio buscam organizar os semestres escolares e escolhas de disciplinas de forma que consigam se formar no menor tempo possível. Apenas alunos calouros não fazem escolhas de disciplinas (e respectivos créditos). A partir do segundo período de seu curso, cada aluno da PUC-Rio deve escolher quais disciplinas para o novo período escolar pretende cursar.

As decisões sobre a escolha de disciplinas são baseadas na oferta de disciplinas daquele período, nos pré-requisitos e compatibilidade de horários. Há também de se considerar turmas distintas de mesmas disciplinas, horários conflitantes, limite no total de vagas oferecidas por turma e total de créditos permitido por semestre. Há também preferências (ou impedimentos) de caráter pessoal, como é o caso de alunos que têm outras atividades (e.g. estágio em empresas ou laboratórios acadêmicos) em paralelo com suas vida acadêmica. Cabe observar que a PUC-Rio determina prioridades em relação à ocupação de vagas de acordo com uma ordenação interna que pondera basicamente o desempenho acadêmico do aluno (C.R. – coeficiente de rendimento), a quantidade de créditos já obtidos, consequentemente, o tempo estimado para completar os créditos necessários para concluir o curso.

A PUC-Rio requisita, a cada semestre, que os alunos preparem uma solicitação de matrícula com três opções contendo o conjunto de disciplinas e turmas que pretende cursar no semestre seguinte. Esta escolha é chamada de grade: disciplinas, dias e horários escolhidos em função da lista de turmas oferecidas pela Universidade para o período seguinte. Há restrições de montagem da grade, como por exemplo, um valor máximo de horas (ou créditos) semanais, turma de matérias distintas par a par; e turmas cujos horários não conflitem entre si, ou seja, que tenham os horários disjuntos, também par a par.

Caso a primeira opção de um aluno não seja totalmente atendida, por exemplo, por falta de vaga, o sistema de matrícula tenta incluir as disciplinas escolhidas em segunda ou terceira opção, buscando atender os pedidos de um aluno. É fato que muitos alunos não se dão por satisfeitos pois muitas disciplinas solicitadas ficam de fora da seleção final de disciplinas efetivamente matriculadas inicialmente. Sem o apoio de uma ferramenta adequada, os ajustes de matrícula em disciplinas, realizados a cada início de semestre escolar, se tornaram cada vez mais frequentes. Entre outros motivos, uma das razões que fazem com que os alunos não fiquem satisfeitos com sua grade de disciplinas escolar diz respeito às decisões equivocadas no momento de solicitação e da montagem da grade de opções.

Consequentemente, a PUC-Rio deve alocar mais recursos para tentar resolver a situação individual de cada aluno pós-solicitação, os alunos precisam comparecer, muitas das vezes, pessoalmente para tentar corrigir suas grades e todo o processo fica mais complicado, e com resultados muitas das vezes indesejáveis. Implicando assim em perdas para ambas as partes.

% \begin{figure}
%   \includegraphics*[width=\linewidth]{ctor4_none.eps}
%   \caption{Uma figura}
% \end{figure}



\chapter{Estado da Arte}

Para auxiliar os alunos na montagem da grade há um grande esforço por parte da PUC-Rio na orientação de como deve ser feito o procedimento de solicitação de matrícula. Este, por sua vez, deve ser realizado através do site da universidade, por meio de uma interface que dá margem à diversas interpretações quanto ao seu funcionamento.

Muitos alunos, por sua vez, buscam se informar e combinar com seus colegas para montar a melhor grade possível. Montando planilhas e fazendo simulações do que poderia acontecer caso suas solicitações não fossem atendidas. Infelizmente esta abordagem implica em uma quantidade de trabalho muito grande, e em cima de interpretações muitas das vezes incorretas de como o processo deve ser feito.

Diversas críticas e discussões acerca do modelo como um todo são feitas. No entanto, nenhuma solução concreta havia sido prosposta. Dessa maneira, o problema persiste.



\chapter{Proposta e Objetivos do Trabalho}

descrição da solução proposta
objetivos específicos a serem alcançados, tendo em vista a definição do problema e os trabalhos relacionados.
    escopo do sistema desejado
    usuarios/programadores e situacoes que se deseja apoiar
    o que se busca avançar com relacao ao estado da arte
    caso não esteja fazendo uma monografia, ele devera elencar todos os itens que serao efetivamente implementados. Ou seja, ele devera ressaltar nesta secao o que realmente sera apresentado para a banca durante a sua apresentacao.


\chapter{Plano de A\c{c}\~ao}

Deve ser destacado que o sistema proposto já foi implementado e usado por cerca de 3000 pessoas nos períodos passados. Dessa maneira, o plano de ação se resume à uma extensa documentação do que foi feito para que, algum dia, alguém possa colocar o projeto para frente utilizando todo o conhecimento adquirido.

Algumas outras funcionalidades como estatísticas de uso e exportação dos dados gerados pelos usuários será implementado na segunda parte do projeto. A primeira é muito interessantes do ponto de vista daqueles que organizam o processo de matrícula. A segunda aos alunos, que não precisariam entrar manualmente com os dados no site da PUC-Rio.

Para que os devidos resultados sejam alcançados será priorizada a documentação. Começando pelo ambiente de desenvolvimento, passando pelo porquê das ferramentas utilizadas até a arquitetura final do sistema.


\arial
\bibliography{Exemplo}

\normalfont
%\begin{thenotations}
% %------------------------------------------------------------------------------%
% \section*{Simplicial Complex}
% %------------------------------------------------------------------------------%
% %
% \noindent
% \begin{tabular}{ll}
% \mathbb{R} & conjunto dos n\'{u}meros reais \\
% \end{tabular}
%
%\end{thenotations}
%\printindex

%\appendix

%\chapter{Primeiro Ap\^{e}ndice}
%O primeiro ap\^{e}ndice deve vir ap\'{o}s as refer\^{e}ncias bibliogr\'{a}ficas. Depois que voc\^{e} colocar a diretiva ``{$\backslash$}apendix'', todos os %``{$\backslash$}chapter\{\}'' v\~{a}o gerar ap\^{e}ndices.

\end{document}
