\documentclass[graduacao,brazil]{ThesisPUC}


%%%%%%%%%%%%%%%%%%%%%%%%%%%%%%%%%%%%%%%%%%%%%%%%%%%%%%%%%%%%%%%%%%%%%%%%%%%%%%%%

\autor{Julio Ribeiro da Silva}
\autorR{Silva, Julio Ribeiro da}
\orientador{S\'{e}rgio Lifschitz}
\orientadorR{Lifschitz, S\'{e}rgio}

\titulo{PrISMA}
\titulouk{PrISMA}

\subtitulo{Programa de Instru\c{c}\~{a}o \`{a} Solicita\c{c}\~{a}o de Matr\'{i}cula Acad\^{e}mica}
\dia{09} \mes{Abril} \ano{2014}

\cidade{Rio de Janeiro}
\CDD{510}
\departamento{Inform\'atica}
\programa{Engenharia da Computa\c{c}\~{a}o}
\centro{Centro T\'{e}cnico Cient\'{i}fico}
\universidade{Pontif\'{i}cia Universidade Cat\'{o}lica do Rio de Janeiro}
\uni{PUC--Rio}
\course{Engenharia da Computa\c{c}\~{a}o}
\diploma{Bacharel em Engenharia da Computa\c{c}\~{a}o}

%%%%%%%%%%%%%%%%%%%%%%%%%%%%%%%%%%%%%%%%%%%%%%%%%%%%%%%%%%%%%%%%%%%%%%%%%%%%%%%%

% N\~{a}o precisa preencher se for um Projeto Final %
%
% \banca{
%   \membrodabanca{Luis Carlos Pacheco R. Velho}{IMPA}
%   \membrodabanca{Jorge Stolfi}{UNICAMP}
%   \coordenador{Ney Augusto Dumont}
% }
%

%%%%%%%%%%%%%%%%%%%%%%%%%%%%%%%%%%%%%%%%%%%%%%%%%%%%%%%%%%%%%%%%%%%%%%%%%%%%%%%%

% N\~{a}o precisa preencher se for um Projeto Final %
%
% \curriculo{%
% Graduou--se em Engenharia na Ecole Polytechnique (Paris, Fran\c{c}a), cursando \'{A}lgebra e Inform\'{a}tica, assim como F\'{i}sica Te\'{o}rica. Especializou--se na Ecole Sup\'{e}rieure des T\'{e}l\'{e}communications (Paris, Fran\c{c}a) em Processamento de Sinais de Voz e Imagens, assim como Organiza\c{c}\~{a}o e Planejamento. Trabalhou junto com a empresa Inventel em sistemas de telecomunica\c{c}\~{o}es sem fil baseados na tecnologia BlueTooth. Desenvolveu junto com os seus orientadores durante o Mestrado ferramentas de topologia computacional.}%
% %


%%%%%%%%%%%%%%%%%%%%%%%%%%%%%%%%%%%%%%%%%%%%%%%%%%%%%%%%%%%%%%%%%%%%%%%%%%%%%%%%

% \epigrafe{%
% Penso, logo mudo de ideia.
% }
% \epigrafeautor{Professor Luiz Fernando Bessa Seibel}
% \epigrafelivro{Regarde}

%%%%%%%%%%%%%%%%%%%%%%%%%%%%%%%%%%%%%%%%%%%%%%%%%%%%%%%%%%%%%%%%%%%%%%%%%%%%%%%%


%\setboolean{hasdedication}{false}
% \agradecimentos{%
% Aos meus orientadores Professores H\'{e}lio Lopes e Geovan Tavares pelo apoio, simpatia de sempre, e incentivo para a realiza\c{c}\~{a}o deste trabalho
% 
% Ao CNPq e \`{a} PUC--Rio, pelos aux\'{i}lios concedidos, sem os quais este trabalho n\~{a}o poderia ter sido realizado.
% 
% \`{A}s minhas av\'{o}s, que sofreram o mais pela saudade devida a minha expatria\c{c}\~{a}o. Aos meus pais, irm\~{a}s e fam\'{i}lia.
% 
% Aos meus colegas da PUC--Rio, quem me fizeram adorar esse lugar.
% 
% Aos professores Marcos da Silvera, Jean--Marie Nicolas e Anne Germa que me ofereceram a oportunidade desta coopera\c{c}\~{a}o.
% 
% Ao pessoal do departamento de Matem\'{a}tica para a ajuda de todos os dias, em particular \`{a} Ana Cristina, Creuza e ao Sinesio.
% }


%%%%%%%%%%%%%%%%%%%%%%%%%%%%%%%%%%%%%%%%%%%%%%%%%%%%%%%%%%%%%%%%%%%%%%%%%%%%%%%%

% \chaves{%
%   \chave{Teoria de Morse}%
%   \chave{Teoria de Forman}%
%   \chave{Topologia Computacional}%
%   \chave{Geometria Computacional}%
%   \chave{Modelagem Geom\'{e}trica}%
%   \chave{Matem\'{a}tica Discreta}%
% }
% 
% \resumo{
% A teoria de Morse \'{e} considerada uma ferramenta matem\'{a}tica importante em aplica\c{c}\~{o}es nas \'{a}reas de topologia computacional, computa\c{c}\~{a}o gr\'{a}fica e modelagem geom\'{e}trica. Ela foi inicialmente formulada para variedades diferenci\'{a}veis. Recentemente, Robin Forman desenvolveu uma vers\~{a}o dessa teoria para estruturas discretas, tais como complexos celulares. E isso permitiu que ela pudesse ser aplicada a outros tipos interessantes de objetos, em particular para malhas.
% 
% Uma vez que uma fun\c{c}\~{a}o de Morse \'{e} definida em uma variedade, informa\c{c}\~{o}es sobre sua topologia podem ser deduzidas atrav\'{e}s de seus elementos cr\'{i}ticos. O objetivo desse trabalho \'{e} apresentar um algoritmo para definir uma fun\c{c}\~{a}o de Morse discreta \'{o}tima para um complexo celular, onde obter o \'{o}timo significa construir uma fun\c{c}\~{a}o que possui o menor n\'{u}mero poss\'{i}vel de elementos cr\'{i}ticos. Aqui foi provado que esse problema \'{e} MAX--SNP dif\'{i}cil. Entretanto, tamb\'{e}m ser\'{a} proposto um algoritmo linear que, para o caso de variedades de dimens\~{a}o 2, \'{e} sempre \'{o}timo.
% 
% Tamb\'{e}m foram provados v\'{a}rios resultados sobre a pr\'{o}pria estrutura das fun\c{c}\~{o}es de Morse discretas. Em particular, uma representa\c{c}\~{a}o equivalente por hiperflorestas \'{e} apresentada. E atrav\'{e}s dessa representa\c{c}\~{a}o, foi desenvolvido um algoritmo para constru\c{c}\~{a}o de fun\c{c}\~{o}es de Morse discretas em complexos celulares com dimens\~{a}o arbitr\'{a}ria. Esse algoritmo \'{e} quadr\'{a}tico no tempo e, apesar de n\~{a}o se poder garantir o resultado \'{o}timo, d\'{a} respostas \'{o}timas na maioria dos casos pr\'{a}ticos.
% }
% 
% 
% %%%%%%%%%%%%%%%%%%%%%%%%%%%%%%%%%%%%%%%%%%%%%%%%%%%%%%%%%%%%%%%%%%%%%%%%%%%%%%%%
% 
% \chavesuk{
%   \chave{Morse Theory}%
%   \chave{Forman Theory}%
%   \chave{Computational Topology}%
%   \chave{Computational Geometry}%
%   \chave{Solid Modeling}%
%   \chave{Discrete Mathematics}%
% }
% 
% \resumouk{%
% Morse theory has been considered a powerful tool in its applications to computational topology, computer graphics and geometric modeling. It was originally formulated for smooth manifolds. Recently, Robin Forman formulated a version of this theory for discrete structures such as cell complexes. It opens up several categories of interesting objects (particularly meshes) to applications of Morse theory.
% 
% Once a Morse function has been defined on a manifold, then information about its topology can be deduced from its critical elements. The purpose of this work is to design an algorithm to define optimal discrete Morse functions on general cell complex, where optimality entails having the least number of critical elements. This problem is proven here to be MAX--SNP hard. However, we provide a linear algorithm that, for the case of 2--manifolds, always reaches optimality.
% 
% Moreover, we proved various results on the structure of a discrete Morse function. In particular, we provide an equivalent representation by hyperforests. From this point of view, we designed a construction of discrete Morse functions for general cell complexes of arbitrary finite dimension. The resulting algorithm is quadratic in time and, although not guaranteed to be optimal, gives optimal answers in most of the practical cases.
% }


%%%%%%%%%%%%%%%%%%%%%%%%%%%%%%%%%%%%%%%%%%%%%%%%%%%%%%%%%%%%%%%%%%%%%%%%%%%%%%%%

\modotabelas{nada} % nada, fig, tab ou figtab

%%%%%%%%%%%%%%%%%%%%%%%%%%%%%%%%%%%%%%%%%%%%%%%%%%%%%%%%%%%%%%%%%%%%%%%%%%%%%%%%

\begin{document}

%%%%%%%%%%%%%%%%%%%%%%%%%%%%%%%%%%%%%%%%%%%%%%%%%%%%%%%%%%%%%%%%%%%%%%%%%%%%%%%%

\chapter{Introdu\c{c}\~ao}

A Pontif\'{i}cia Universidade Cat\'{o}lica do Rio de Janeiro (PUC-Rio) oferece a seus alunos diversos cursos de gradua\c{c}\~{a}o, cujas disciplinas s\~{a}o contabilizadas e disponibilizadas em um sistema de contagem de cr\'{e}ditos. Os alunos precisam conter em seus hist\'{o}ricos escolares um n\'{u}mero m\'{i}nimo de cr\'{e}ditos pertencentes a um determinado curr\'{i}culo de um curso para que obtenham seus diplomas. 

Assim, e ao longo de sua vida universit\'{a}ria, os alunos da PUC-Rio buscam organizar os semestres escolares e escolhas de disciplinas de forma que consigam se formar no menor tempo poss\'{i}vel. Apenas alunos calouros n\~{a}o fazem escolhas de disciplinas (e respectivos cr\'{e}ditos). A partir do segundo per\'{i}odo de seu curso, cada aluno da PUC-Rio deve escolher quais disciplinas para o novo per\'{i}odo escolar pretende cursar.

As decisões sobre a escolha de disciplinas s\~{a}o baseadas na oferta de disciplinas daquele per\'{i}odo, nos pr\'{e}-requisitos e compatibilidade de hor\'{a}rios. H\'{a} tamb\'{e}m de se considerar turmas distintas de mesmas disciplinas, hor\'{a}rios conflitantes, limite no total de vagas oferecidas por turma e total de cr\'{e}ditos permitido por semestre. H\'{a} tamb\'{e}m prefer\^{e}ncias (ou impedimentos) de car\'{a}ter pessoal, como \'{e} o caso de alunos que t\^{e}m outras atividades (e.g. est\'{a}gio em empresas ou laborat\'{o}rios acad\^{e}micos) em paralelo com suas vida acad\^{e}mica. Cabe observar que a PUC-Rio determina prioridades em rela\c{c}\~{a}o \`{a} ocupa\c{c}\~{a}o de vagas de acordo com uma ordena\c{c}\~{a}o interna que pondera basicamente o desempenho acad\^{e}mico do aluno (C.R. – coeficiente de rendimento), a quantidade de cr\'{e}ditos j\'{a} obtidos, consequentemente, o tempo estimado para completar os cr\'{e}ditos necess\'{a}rios para concluir o curso.

A PUC-Rio requisita, a cada semestre, que os alunos preparem uma solicita\c{c}\~{a}o de matr\'{i}cula com tr\^{e}s op\c{c}ões contendo o conjunto de disciplinas e turmas que pretende cursar no semestre seguinte. Esta escolha \'{e} chamada de grade: disciplinas, dias e hor\'{a}rios escolhidos em fun\c{c}\~{a}o da lista de turmas oferecidas pela Universidade para o per\'{i}odo seguinte. H\'{a} restri\c{c}ões de montagem da grade, como por exemplo, um valor m\'{a}ximo de horas (ou cr\'{e}ditos) semanais, turma de mat\'{e}rias distintas par a par; e turmas cujos hor\'{a}rios n\~{a}o conflitem entre si, ou seja, que tenham os hor\'{a}rios disjuntos, tamb\'{e}m par a par.

Caso a primeira op\c{c}\~{a}o de um aluno n\~{a}o seja totalmente atendida, por exemplo, por falta de vaga, o sistema de matr\'{i}cula tenta incluir as disciplinas escolhidas em segunda ou terceira op\c{c}\~{a}o, buscando atender os pedidos de um aluno. \'{e} fato que muitos alunos n\~{a}o se d\~{a}o por satisfeitos pois muitas disciplinas solicitadas ficam de fora da sele\c{c}\~{a}o final de disciplinas efetivamente matriculadas inicialmente. Sem o apoio de uma ferramenta adequada, os ajustes de matr\'{i}cula em disciplinas, realizados a cada in\'{i}cio de semestre escolar, se tornaram cada vez mais frequentes. Entre outros motivos, uma das razões que fazem com que os alunos n\~{a}o fiquem satisfeitos com sua grade de disciplinas escolar diz respeito \`{a}s decisões equivocadas no momento de solicita\c{c}\~{a}o e da montagem da grade de op\c{c}ões.

Consequentemente, a PUC-Rio deve alocar mais recursos para tentar resolver a situa\c{c}\~{a}o individual de cada aluno p\'{o}s-solicita\c{c}\~{a}o, os alunos precisam comparecer, muitas das vezes, pessoalmente para tentar corrigir suas grades e todo o processo fica mais complicado, e com resultados muitas das vezes indesej\'{a}veis. Implicando assim em perdas para ambas as partes.

% \begin{figure}
%   \includegraphics*[width=\linewidth]{ctor4_none.eps}
%   \caption{Uma figura}
% \end{figure}

%%%%%%%%%%%%%%%%%%%%%%%%%%%%%%%%%%%%%%%%%%%%%%%%%%%%%%%%%%%%%%%%%%%%%%%%%%%%%%%%

\chapter{Estado da Arte}

Para auxiliar os alunos na montagem da grade h\'{a} um grande esfor\c{c}o por parte da PUC-Rio na orienta\c{c}\~{a}o de como deve ser feito o procedimento de solicita\c{c}\~{a}o de matr\'{i}cula. Este, por sua vez, deve ser realizado atrav\'{e}s do site da universidade, por meio de uma interface que d\'{a} margem \`{a} diversas interpreta\c{c}ões quanto ao seu funcionamento.

Muitos alunos, por sua vez, buscam se informar e combinar com seus colegas para montar a melhor grade poss\'{i}vel. Montando planilhas e fazendo simula\c{c}ões do que poderia acontecer caso suas solicita\c{c}ões n\~{a}o fossem atendidas. Infelizmente esta abordagem implica em uma quantidade de trabalho muito grande, e em cima de interpreta\c{c}ões muitas das vezes incorretas de como o processo deve ser feito.

Diversas cr\'{i}ticas e discussões acerca do modelo como um todo s\~{a}o feitas. No entanto, nenhuma solu\c{c}\~{a}o concreta havia sido prosposta ainda. De acordo com o apresentado at\'{e} o momento, o problema persiste.

%%%%%%%%%%%%%%%%%%%%%%%%%%%%%%%%%%%%%%%%%%%%%%%%%%%%%%%%%%%%%%%%%%%%%%%%%%%%%%%%

\chapter{Proposta e Objetivos do Trabalho}

Propõe-se ent\~{a}o o sistema PrISMA (Programa de Instru\c{c}\~{a}o \`{a} Solicita\c{c}\~{a}o de Matr\'{i}cula acad\^{e}mica. Trata-se de uma webpage que reune todas as funcionalidades necess\'{a}rias para auxiliar os alunos no processo de matr\'{i}cula.

O objetivo do sistema \'{e} apresentar uma interface muito poderosa, reunindo todos os dados necess\'{a}rios \`{a} solicita\c{c}\~{a}o, e ao mesmo tempo intuitiva, de forma que os alunos n\~{a}o fiquem perdidos durante o processo. No entanto, o processo como um todo exige o acesso \`{a} muita informa\c{c}\~{a}o ao mesmo tempo, tornando o desenvolvimento do projeto um grande desafio.

O b\'{a}sico, para os alunos, seria reunir num mesmo ambiente seu Falta Cursar, disciplinas oferecidas pelo Micro Hor\'{a}rio e as disciplinas j\'{a} selecionadas para a grade do pr\'{o}ximo per\'{i}odo. Assim como uma visualiza\c{c}\~{a}o daquilo de j\'{a} foi selecionado no formato de tabela de hor\'{a}rios semanal. Dessa maneira, n\~{a}o seria mais necess\'{a}rio se perder no meio de diversas aplica\c{c}ões em busca de todas essas informa\c{c}ões. Tudo estaria reunido em um \'{u}nico lugar.

Al\'{e}m da preocupa\c{c}\~{a}o com a nevega\c{c}\~{a}o, existe tamb\'{e}m a preocupa\c{c}\~{a}o com a corretude da grade gerada. Por especifica\c{c}\~{a}o todas (ou pelo menos a maioria) das regras que impedem a solicita\c{c}\~{a}o de disciplinas devem ser implementadas. Sendo a principal delas a regra de pr\'{e}-requisitos. Para tal se faz necess\'{a}rio do hist\'{o}rico de disciplinas cursadas dos alunos, e do seu C.R. acumulado. Dessa maneira, conforme o aluno for utilizando e selecionando as disciplinas a serem cursadas, o sistema realiza a verifica\c{c}\~{a}o e informa o aluno em caso de algum problema.

Buscando diminuir a quantidade de dados apresentados, ao inv\'{e}s de exibir as disciplinas do Falta Cursar do aluno, \'{e} pr\'{e}computado o conjunto de disciplinas chamado de Pode Cursar. Este subconjunto cont\'{e}m apenas as disciplinas que n\~{a}o est\~{a}o presas por nenhum pr\'{e}-requisito. 

Olhando do ponto de vista das multiplas op\c{c}ões no campo de solicita\c{c}\~{a}o, identifica-se uma grande incompreens\~{a}o por parte dos alunos dessa funcionalidade no sistema da PUC. Essa falta de clareza se trata de uma das principais fontes de erros dos alunos na hora de preencher a solicita\c{c}\~{a}o. Por se tratar de um ponto cr\'{i}tico, propõe-se uma funcionalidade que, passo a passo, mostre ao usu\'{a}rio o que aconteceria caso alguma disciplina, de algum op\c{c}\~{a}o fosse rejeitada, e que impacto isso teria no restante da solicita\c{c}\~{a}o.

Ap\'{o}s terem suas grades montadas, deve-se preencher manualmente no site da PUC-Rio os dados gerados, visto que se trata de um sistema independente. Uma solu\c{c}\~{a}o para isto seria a defini\c{c}\~{a}o de um padr\~{a}o de arquivo exportado pelo PrISMA que fosse aceito pelo PUC Online (setor respons\'{a}vel pelo sistema da PUC-Rio). Entretanto, por se tratar de procedimentos burocr\'{a}ticos, n\~{a}o ser\'{a} de fato implementado pela PUC. Apesar de que o arquivo ainda assim possa ser gerado.

Para aqueles que oferecem as disciplinas \'{e} interessante ter dados estat\'{i}sticos de uso do sistema por parte dos alunos. Apresenta\c{c}ões de gr\'{a}ficos de lota\c{c}\~{a}o das turmas e como se comportam os alunos ao realizar o procedimento de escolha. Dessa maneira pode-se aprender com a necessidade dos alunos.

%%%%%%%%%%%%%%%%%%%%%%%%%%%%%%%%%%%%%%%%%%%%%%%%%%%%%%%%%%%%%%%%%%%%%%%%%%%%%%%%

\chapter{Plano de A\c{c}\~ao}

Deve ser destacado que o sistema proposto j\'{a} foi implementado e usado por cerca de 3000 pessoas nos per\'{i}odos passados. Dessa maneira, o plano de a\c{c}\~{a}o se resume \`{a} uma extensa documenta\c{c}\~{a}o do que foi feito para que, algum dia, algu\'{e}m possa colocar o projeto para frente utilizando todo o conhecimento adquirido.

Algumas outras funcionalidades como estat\'{i}sticas de uso e exporta\c{c}\~{a}o dos dados gerados pelos usu\'{a}rios ser\'{a} implementado na segunda parte do projeto. A primeira \'{e} muito interessantes do ponto de vista daqueles que organizam o processo de matr\'{i}cula. A segunda aos alunos, que n\~{a}o precisariam entrar manualmente com os dados no site da PUC-Rio.

Para que os devidos resultados sejam alcan\c{c}ados ser\'{a} priorizada a documenta\c{c}\~{a}o. Come\c{c}ando pelo ambiente de desenvolvimento, passando pelo porqu\^{e} das ferramentas utilizadas at\'{e} a arquitetura final do sistema.

%%%%%%%%%%%%%%%%%%%%%%%%%%%%%%%%%%%%%%%%%%%%%%%%%%%%%%%%%%%%%%%%%%%%%%%%%%%%%%%%

\arial
\nocite{*} %% <<<------------ remover!!!
\bibliography{proposta}

\normalfont
%\begin{thenotations}
% %------------------------------------------------------------------------------%
% \section*{Simplicial Complex}
% %------------------------------------------------------------------------------%
% %
% \noindent
% \begin{tabular}{ll}
% \mathbb{R} & conjunto dos n\'{u}meros reais \\
% \end{tabular}
%
%\end{thenotations}
%\printindex

%\appendix

%\chapter{Primeiro Ap\^{e}ndice}
%O primeiro ap\^{e}ndice deve vir ap\'{o}s as refer\^{e}ncias bibliogr\'{a}ficas. Depois que voc\^{e} colocar a diretiva ``{$\backslash$}apendix'', todos os %``{$\backslash$}chapter\{\}'' v\~{a}o gerar ap\^{e}ndices.

\end{document}
